%!TEX output_directory = aux

\documentclass[11pt]{article}

\usepackage[english]{babel}

% ---- FONT & MICROTYPOGRAPHY ----
\usepackage[utf8]{inputenc}
\usepackage[T1]{fontenc}
\usepackage{microtype}
\usepackage{moresize}

% ---- FORMATTING ----
\usepackage{csquotes,textcase,xspace}
\usepackage[normalem]{ulem}

% ---- PAGE LAYOUT ----
\usepackage{geometry}
\geometry{top=2.5cm,bottom=2cm,inner=2cm,outer=2cm,footnotesep=7mm plus 4pt minus 4pt}
\usepackage{setspace}
\setstretch{1.1}

% ---- GRAPHIQUE ----
\usepackage{graphicx}
\usepackage{xcolor}
\usepackage[font=small,labelfont=bf,labelsep=space]{caption}
\usepackage{subfigure}
\captionsetup{width=0.9\textwidth,font={small,stretch=1.1}}
\addto\captionsenglish{\renewcommand{\figurename}{Fig.}}
\addto\captionsenglish{\renewcommand{\tablename}{Tab.}}
\definecolor{JoliBleu}{rgb}{0,0.55,0.55}
\definecolor{JoliVert}{rgb}{0.15,0.6,0}
\definecolor{JoliRouge}{rgb}{0.86,0.08,0}
\definecolor{JoliJaune}{rgb}{1,0.75,0}
\definecolor{JoliGris}{rgb}{0.52,0.52,0.51}
\definecolor{myblue}{RGB}{26, 77, 116}
\definecolor{myorange}{RGB}{181, 116, 30}
\definecolor{mydarkorange}{RGB}{166, 88, 0}
\definecolor{mygreen}{RGB}{21, 124, 80}
\definecolor{myblack}{RGB}{43, 65, 82}
\definecolor{myred}{rgb}{0.5, 0.0, 0.13}

% ---- SECTIONING ----
\usepackage{titlesec}
\titleformat{\section}[block]{\Large\boldmath\bfseries}{\thesection}{1em}{}
\titleformat{\subsection}[block]{\large\boldmath\bfseries}{\thesubsection}{0.5em}{}
\usepackage{appendix}
\renewcommand{\setthesection}{\Alph{section}}
\renewcommand{\restoreapp}{}
\makeatletter
\renewcommand{\theequation}{\thesection.\arabic{equation}}
\@addtoreset{equation}{section}
\makeatother

% ---- FOOTERS HEADERS ----
\usepackage[bottom]{footmisc}
\usepackage{fancyhdr}

% ---- TABLE OF CONTENTS ----
\usepackage{titletoc}
\setcounter{tocdepth}{3}

% ---- BIBLIOGRAPHY ----
\usepackage[nosort]{cite}
\bibliographystyle{JHEP}
\newcommand{\eprint}[1]{{\href{http://arxiv.org/abs/#1}{\texttt{[#1]}}}}
\newcommand{\eprintN}[1]{{\href{http://arxiv.org/abs/#1}{\texttt{#1 [hep-th]}}}}
\newcommand{\doi}[2]{\href{http://dx.doi.org/#2}{#1}}

% ---- HYPER REF ----
\usepackage{hyperref}
\hypersetup{colorlinks=true,
        pdfstartview=FitV,
        linkcolor= mydarkorange,
        citecolor= mydarkorange,
        urlcolor= JoliGris!60!black,
        hypertexnames=false,
        linktoc=page}

% ---- TIKZ ----
\usepackage{tikz}
\usetikzlibrary{calc}
\newcommand{\repone}[1]{%
    \tikz[baseline=(ref.base)]{
        \node[inner sep=0pt,outer sep=0pt] (ref) {$#1$};
        \draw ($(ref)+(-0.1,0.4)$) circle (2.5pt);
    }
  }
\newcommand{\reptwo}[1]{%
    \tikz[baseline=(ref.base)]{
        \node[inner sep=0pt,outer sep=0pt] (ref) {$#1$};
        \draw ($(ref)+(-0.2,0.3)$) rectangle +(0.17,0.17);
    }
  }
\newcommand{\repthree}[1]{%
    \tikz[baseline=(ref.base)]{
        \node[inner sep=0pt,outer sep=0pt] (ref) {$#1$};
        \draw ($(ref)+(-0.35,0.35)$) -- +(0.1,0.1) -- +(0.2,0) -- +(0.1,-0.1) -- +(0,0);
    }
  }
\newcommand{\repfour}[1]{%
    \tikz[baseline=(ref.base)]{
        \node[inner sep=0pt,outer sep=0pt] (ref) {$#1$};
        \draw ($(ref)+(-0.2,0.3)$) -- +(0.2,0) -- +(0.1,0.15) -- +(0,0);
    }
  }
\newcommand{\repfive}[1]{%
    \tikz[baseline=(ref.base)]{
        \node[inner sep=0pt,outer sep=0pt] (ref) {$#1$};
        \draw ($(ref)+(-0.2,0.4)$) -- +(0.2,0) -- +(0.1,-0.15) -- +(0,0);
    }
  }  

% ---- MATHS ----
\usepackage{amsmath,amssymb,amsfonts,dsfont}
\usepackage{mathrsfs}
\usepackage{physics}
\usepackage{ytableau}
\ytableausetup{boxsize=1.1em,centertableaux}
\usepackage{stmaryrd}
\usepackage{nicefrac}
\allowdisplaybreaks[1]
% \usepackage{bbold}
\usepackage{cases}
\usepackage{bm}
\usepackage{bbm}

% ---- TABLES ----
\usepackage{multirow}
\usepackage{booktabs}
\usepackage{pdflscape}
\usepackage{array}
\usepackage{arydshln}

% ---- ENUMERATION ----
\usepackage[shortlabels]{enumitem}

% ---- MATHS COMMANDS ----
\newcommand{\A}{\ensuremath{\mathcal{A}}\xspace}
\newcommand{\F}{\ensuremath{\mathcal{F}}\xspace}
\renewcommand{\H}{\ensuremath{\mathcal{H}}\xspace}
\newcommand{\M}{\ensuremath{\mathcal{M}}\xspace}
\renewcommand{\P}{\ensuremath{\mathcal{P}}\xspace}
\newcommand{\J}{\ensuremath{\mathcal{J}}\xspace}
\renewcommand{\d}{\ensuremath{\mathrm{d}}\xspace}
\renewcommand{\H}{\ensuremath{\mathcal{H}}\xspace}
\newcommand{\SO}{\ensuremath{\mathrm{SO}}\xspace}
\renewcommand{\O}{\ensuremath{\mathrm{O}}\xspace}
\newcommand{\SL}{\ensuremath{\mathrm{SL}}\xspace}
\newcommand{\SU}{\ensuremath{\mathrm{SU}}\xspace}
\newcommand{\Odd}{\ensuremath{\mathrm{O}(d,d)}\xspace}
\newcommand{\odd}{\ensuremath{\mathfrak{o}(d,d)}\xspace}
\renewcommand{\Tr}[1]{\ensuremath{\mathrm{Tr}\left(#1\right)}\xspace}
\newcommand{\vol}{{\,\rm vol}}
\def\sst#1{{\scriptscriptstyle #1}}


\def\0{{\sst{(0)}}}
\def\1{{\sst{(1)}}}
\def\2{{\sst{(2)}}}
\def\3{{\sst{(3)}}}
\def\4{{\sst{(4)}}}
\def\5{{\sst{(5)}}}
\def\6{{\sst{(6)}}}
\def\7{{\sst{(7)}}}
\def\8{{\sst{(8)}}}

\newcommand{\be}{\begin{equation}}
\newcommand{\ee}{\end{equation}}

% ---- COMMENTS ----
\newcommand{\ce}[1]{\marginpar{\parbox{\marginparwidth}{\boldmath $\Longleftarrow$}}{\boldmath\bfseries (ce: #1)}}
\newcommand{\gl}[1]{\marginpar{\parbox{\marginparwidth}{\boldmath $\Longleftarrow$}}{\boldmath\bfseries (gl: #1)}}

% ---- TITLE PAGE ----
\usepackage[affil-it]{authblk}
\def\preprint{}

\makeatletter
\def\@maketitle{%
  \newpage
  \null\hfill\texttt{\preprint}
  \vskip 4em%
  \begin{center}%
  \let \footnote \thanks
    {\LARGE\bfseries \@title \par}%
    \vskip 2.5em%
    {\large
      \lineskip .5em%
      \begin{center}
        \begin{minipage}{0.95\textwidth}
            \begin{tabular}[t]{c}%
            \@author
            \end{tabular}
        \end{minipage}    
      \end{center}\par}%
    \vskip 1em%
    {\large \@date}%
  \end{center}%
  \par
  \vskip 1.5em}
\makeatother

\renewcommand\Authands{ and }

\title{Notes: cubic couplings in $3d$}
\author{}



%%%%%%%%%%%%%%%%%%%%%%%%%%%%%%%%%%
%%%%%%%%%%%%%%%%%%%%%%%%%%%%%%%%%%


\begin{document}

\maketitle

\section*{Scalar fields}
The scalar Kaluza-Klein fluctuations of the $\SO(8,4+m)$ exceptional field theory~\cite{Hohm:2017wtr} are labelled by pair of indices $\Phi^{\alpha,\Sigma}$~\cite{Eloy:2020uix}, with $\alpha$ the index of non-compact generators of $\SO(8,4+m)$ (or equivalently the fields within the $3d$ truncation) and $\Sigma$ labelling the scalar harmonic ${\cal Y}^{\Sigma}$ on $S^{3}$. They transform in the symmetric vector representation $\big(\nicefrac{n}{2},\nicefrac{n}{2};0,0\big)$ of $\SO(4)_{\rm gauge} \times \SO(4)_{\rm global} \times \SO(m)$. Under
\begin{equation}
	\SO(8,4+m) \longrightarrow \SO(4)_{\rm gauge} \times \SO(4)_{\rm global} \times \SO(m),
\end{equation}
the non-compact generators $t_{\alpha}$ of $\SO(8,4+m)$ decompose as
\begin{equation}
	\begin{aligned}
		t_{\alpha} &\longrightarrow \big(0,0;0,0\big) \oplus \big(1,0;0,0\big) \oplus \big(0,1;0,0\big) \oplus \big(1,1;0,0\big) \\
		& \qquad \oplus \big(\nicefrac{1}{2},\nicefrac{1}{2};\nicefrac{1}{2},\nicefrac{1}{2}\big) \oplus \big(\nicefrac{1}{2},\nicefrac{1}{2};0,0\big)^{(m)} \oplus \big(0,0;\nicefrac{1}{2},\nicefrac{1}{2}\big)^{(m)},
	\end{aligned}
\end{equation}
where the exponent ${}^{(m)}$ indicates representations that transforms as vectors under $\SO(m)$. The other representations are $\SO(m)$ scalars. Then, the fluctuations $\Phi^{\alpha,\Sigma}$ are made of the following representations:
\begin{equation} \label{eq:scalarsrep}
	\begin{aligned}
		\alpha \otimes \Sigma &\longrightarrow \Big[\big(0,0;0,0\big) \oplus \big(1,0;0,0\big) \oplus \big(0,1;0,0\big) \oplus \big(1,1;0,0\big) \\
		& \qquad \oplus \big(\nicefrac{1}{2},\nicefrac{1}{2};\nicefrac{1}{2},\nicefrac{1}{2}\big) \oplus \big(\nicefrac{1}{2},\nicefrac{1}{2};0,0\big)^{(m)} \oplus \big(0,0;\nicefrac{1}{2},\nicefrac{1}{2}\big)^{(m)}\Big] \otimes \big(\nicefrac{n}{2},\nicefrac{n}{2};0,0\big) \\
		&\longrightarrow \Big[\big(\nicefrac{n}{2},\nicefrac{n}{2};0,0\big)\Big] \\
		&\qquad \oplus \Big[\big(\nicefrac{(n-2)}{2},\nicefrac{n}{2};0,0\big) \oplus \big(\nicefrac{n}{2},\nicefrac{n}{2};0,0\big) \oplus \big(\nicefrac{(n+2)}{2},\nicefrac{n}{2};0,0\big)\Big]\\
		& \qquad\oplus \Big[\big(\nicefrac{n}{2},\nicefrac{(n-2)}{2};0,0\big) \oplus \big(\nicefrac{n}{2},\nicefrac{n}{2};0,0\big) \oplus \big(\nicefrac{n}{2},\nicefrac{(n+2)}{2};0,0\big)\Big] \\
		& \qquad\oplus \Big[\big(\nicefrac{(n-2)}{2},\nicefrac{(n-2)}{2};0,0\big) \oplus \big(\nicefrac{(n-2)}{2},\nicefrac{n}{2};0,0\big) \oplus \big(\nicefrac{(n-2)}{2},\nicefrac{(n+2)}{2};0,0\big) \\
		&\qquad \qquad \oplus \big(\nicefrac{n}{2},\nicefrac{(n-2)}{2};0,0\big) \oplus \big(\nicefrac{n}{2},\nicefrac{n}{2};0,0\big) \oplus \big(\nicefrac{n}{2},\nicefrac{(n+2)}{2};0,0\big) \\
		&\qquad \qquad \oplus \big(\nicefrac{(n+2)}{2},\nicefrac{(n-2)}{2};0,0\big) \oplus \big(\nicefrac{(n+2)}{2},\nicefrac{n}{2};0,0\big) \oplus \big(\nicefrac{(n+2)}{2},\nicefrac{(n+2)}{2};0,0\big)\Big] \\
		& \qquad \oplus \Big[\big(\nicefrac{(n-1)}{2},\nicefrac{(n-1)}{2};\nicefrac{1}{2},\nicefrac{1}{2}\big) \oplus \big(\nicefrac{(n-1)}{2},\nicefrac{(n+1)}{2};\nicefrac{1}{2},\nicefrac{1}{2}\big) \\
		& \qquad \qquad \oplus \big(\nicefrac{(n+1)}{2},\nicefrac{(n-1)}{2};\nicefrac{1}{2},\nicefrac{1}{2}\big) \oplus \big(\nicefrac{(n+1)}{2},\nicefrac{(n+1)}{2};\nicefrac{1}{2},\nicefrac{1}{2}\big)\Big]\\
		& \qquad \oplus \Big[\big(\nicefrac{(n-1)}{2},\nicefrac{(n-1)}{2};0,0\big)^{(m)} \oplus \big(\nicefrac{(n-1)}{2},\nicefrac{(n+1)}{2};0,0\big)^{(m)} \\
		& \qquad \qquad \oplus \big(\nicefrac{(n+1)}{2},\nicefrac{(n-1)}{2};0,0\big)^{(m)} \oplus \big(\nicefrac{(n+1)}{2},\nicefrac{(n+1)}{2};0,0\big)^{(m)}\Big] \\
		& \qquad \oplus \Big[\big(\nicefrac{n}{2},\nicefrac{n}{2};\nicefrac{1}{2},\nicefrac{1}{2}\big)^{(m)}\Big].
	\end{aligned}
\end{equation}
This corresponds to the scalars fields and Goldstone modes belonging to the level $n$ in the Kaluza-Klein tower. The supermultiplets at level $n$ are~\cite{Deger:1998nm,deBoer:1998kjm,Eloy:2020uix} (see tab.~\ref{tab:multi20} for the notations)
% \begin{equation}
% 	{\cal S}^{(n)}_{(2,0)} = [\boldsymbol{n+1},\boldsymbol{n+1}]_{\rm s}+[\boldsymbol{n+3},\boldsymbol{n+3}]_{\rm s}+[\boldsymbol{n+2},\boldsymbol{n+2}]_{\rm s}^{(m)}
%   +[\boldsymbol{n+1},\boldsymbol{n+3}]_{\rm s}+[\boldsymbol{n+3},\boldsymbol{n+1}]_{\rm s}.
% \end{equation}
\begin{equation}
	{\cal S}^{(n)}_{(2,0)} = \repone{[\boldsymbol{n+1},\boldsymbol{n+1}]_{\rm s}}+\reptwo{[\boldsymbol{n+3},\boldsymbol{n+3}]_{\rm s}}+\repthree{[\boldsymbol{n+2},\boldsymbol{n+2}]_{\rm s}^{(m)}}
  +\repfour{[\boldsymbol{n+1},\boldsymbol{n+3}]_{\rm s}}+\repfive{[\boldsymbol{n+3},\boldsymbol{n+1}]_{\rm s}}.
\end{equation}
In this equation, we have assigned a geometric shape to each of these supermultiplets. We use in the following these shapes to identify the multiplet to which each representation in eq.~\eqref{eq:scalarsrep} belongs:
\begin{equation}
	\begin{aligned}
		\alpha \otimes \Sigma 
		&\longrightarrow \Big[\dashuline{\big(\nicefrac{n}{2},\nicefrac{n}{2};0,0\big)}\Big] \\
		&\qquad \oplus \Big[\underline{\big(\nicefrac{(n-2)}{2},\nicefrac{n}{2};0,0\big)} \oplus \dashuline{\big(\nicefrac{n}{2},\nicefrac{n}{2};0,0\big)} \oplus \underline{\big(\nicefrac{(n+2)}{2},\nicefrac{n}{2};0,0\big)}\Big]\\
		& \qquad\oplus \Big[\underline{\big(\nicefrac{n}{2},\nicefrac{(n-2)}{2};0,0\big)} \oplus \dashuline{\big(\nicefrac{n}{2},\nicefrac{n}{2};0,0\big)} \oplus \underline{\big(\nicefrac{n}{2},\nicefrac{(n+2)}{2};0,0\big)}\Big] \\
		& \qquad\oplus \Big[\repone{\big(\nicefrac{(n-2)}{2},\nicefrac{(n-2)}{2};0,0\big)} \oplus \underline{\big(\nicefrac{(n-2)}{2},\nicefrac{n}{2};0,0\big)} \oplus \repfour{\big(\nicefrac{(n-2)}{2},\nicefrac{(n+2)}{2};0,0\big)} \\
		&\qquad \qquad \oplus \underline{\big(\nicefrac{n}{2},\nicefrac{(n-2)}{2};0,0\big)} \oplus \dashuline{\big(\nicefrac{n}{2},\nicefrac{n}{2};0,0\big)} \oplus \underline{\big(\nicefrac{n}{2},\nicefrac{(n+2)}{2};0,0\big)} \\
		&\qquad \qquad \oplus \repfive{\big(\nicefrac{(n+2)}{2},\nicefrac{(n-2)}{2};0,0\big)} \oplus \underline{\big(\nicefrac{(n+2)}{2},\nicefrac{n}{2};0,0\big)} \oplus \reptwo{\big(\nicefrac{(n+2)}{2},\nicefrac{(n+2)}{2};0,0\big)}\Big] \\
		& \qquad \oplus \Big[\repone{\big(\nicefrac{(n-1)}{2},\nicefrac{(n-1)}{2};\nicefrac{1}{2},\nicefrac{1}{2}\big)} \oplus \underline{\repfour{\big(\nicefrac{(n-1)}{2},\nicefrac{(n+1)}{2};\nicefrac{1}{2},\nicefrac{1}{2}\big)}} \\
		& \qquad \qquad \oplus \underline{\repfive{\big(\nicefrac{(n+1)}{2},\nicefrac{(n-1)}{2};\nicefrac{1}{2},\nicefrac{1}{2}\big)}} \oplus \reptwo{\big(\nicefrac{(n+1)}{2},\nicefrac{(n+1)}{2};\nicefrac{1}{2},\nicefrac{1}{2}\big)}\Big]\\
		& \qquad \oplus \Big[\repthree{\big(\nicefrac{(n-1)}{2},\nicefrac{(n-1)}{2};0,0\big)^{(m)}} \oplus \repthree{\underline{\big(\nicefrac{(n-1)}{2},\nicefrac{(n+1)}{2};0,0\big)}^{(m)}} \\
		& \qquad \qquad \oplus \repthree{\underline{\big(\nicefrac{(n+1)}{2},\nicefrac{(n-1)}{2};0,0\big)}^{(m)}} \oplus \repthree{\big(\nicefrac{(n+1)}{2},\nicefrac{(n+1)}{2};0,0\big)^{(m)}}\Big] \\
		& \qquad \oplus \Big[\repthree{\big(\nicefrac{n}{2},\nicefrac{n}{2};\nicefrac{1}{2},\nicefrac{1}{2}\big)^{(m)}}\Big].
	\end{aligned}
\end{equation}
Representations to which no shape has been assigned may belong to multiple multiplets \ce{This needs to be fixed}. Underlined representations are Goldstone modes (or potential Goldstone modes if the line is dashed and not solid).

\begin{table}[t!]
\renewcommand{\arraystretch}{1.5}
  \centering
  \begin{tabular}{cccccc}
    $\Delta_{\rm L}$ & $\Delta_{\rm R}$ & $\Delta$ & $s$ & $\SO(4)_{\rm gauge}$ & $\SO(4)_{\rm global}$ \\ \hline\hline
    \multicolumn{6}{c}{\bfseries Spin-1 multiplet $\boldsymbol{[k+1,k+1]_{\rm s}}$} \\ \hline
    $k/2$ & $k/2$ & $k$ & $0$ & $\big(k/2,k/2\big)$ & $\big(0,0\big)$ \\
    $k/2$ & $(k+1)/2$ & $k+1/2$ & $1/2$ & $\big(k/2,(k-1)/2\big)$ & $\big(0,1/2\big)$ \\
    $(k+1)/2$ & $k/2$ & $k+1/2$ & $-1/2$ & $\big((k-1)/2,k/2\big)$ & $\big(1/2,0\big)$ \\
    $(k+1)/2$ & $(k+1)/2$ & $k+1$ & $0$ & $\big((k-1)/2,(k-1)/2\big)$ & $\big(1/2,1/2\big)$ \\
    $k/2$ & $(k+2)/2$ & $k+1$ & $1$ & $\big(k/2,(k-2)/2\big)$ & $\big(0,0\big)$ \\
    $(k+2)/2$ & $k/2$ & $k+1$ & $-1$ & $\big((k-2)/2,k/2\big)$ & $\big(0,0\big)$ \\
    $(k+1)/2$ & $(k+2)/2$ & $k+3/2$ & $1/2$ & $\big((k-1)/2,(k-2)/2\big)$ & $\big(1/2,0\big)$ \\
    $(k+2)/2$ & $(k+1)/2$ & $k+3/2$ & $-1/2$ & $\big((k-2)/2,(k-1)/2\big)$ & $\big(0,1/2\big)$ \\
    $(k+2)/2$ & $(k+2)/2$ & $k+2$ & $0$ & $\big((k-2)/2,(k-2)/2\big)$ & $\big(0,0\big)$ \\ \hline
    \multicolumn{6}{c}{\bfseries Spin-2 multiplet $\boldsymbol{[p,p+2]_{\rm s}}$} \\ \hline
    $(p-1)/2$ & $(p+1)/2$ & $p$ & $1$ & $\big((p-1)/2,(p+1)/2\big)$ & $\big(0,0\big)$ \\
    $(p-1)/2$ & $(p+2)/2$ & $p+1/2$ & $3/2$ & $\big((p-1)/2,p/2\big)$ & $\big(0,1/2\big)$ \\
    $p/2$ & $(p+1)/2$ & $p+1/2$ & $1/2$ & $\big((p-2)/2,(p+1)/2\big)$ & $\big(1/2,0\big)$ \\
    $p/2$ & $(p+2)/2$ & $p+1$ & $1$ & $\big((p-2)/2,p/2\big)$ & $\big(1/2,1/2\big)$ \\
    $(p-1)/2$ & $(p+3)/2$ & $p+1$ & $2$ & $\big((p-1)/2,(p-1)/2\big)$ & $\big(0,0\big)$ \\
    $(p+1)/2$ & $(p+1)/2$ & $p+1$ & $0$ & $\big((p-3)/2,(p+1)/2\big)$ & $\big(0,0\big)$ \\
    $p/2$ & $(p+3)/2$ & $p+3/2$ & $3/2$ & $\big((p-2)/2,(p-1)/2\big)$ & $\big(1/2,0\big)$ \\
    $(p+1)/2$ & $(p+2)/2$ & $p+3/2$ & $1/2$ & $\big((p-3)/2,p/2\big)$ & $\big(0,1/2\big)$ \\
    $(p+1)/2$ & $(p+3)/2$ & $p+2$ & $1$ & $\big((p-3)/2,(p-1)/2\big)$ & $\big(0,0\big)$
  \end{tabular}
  \caption{Spin-1 $[\boldsymbol{k+1},\boldsymbol{k+1}]_{\rm s}$ and spin-2 $[\boldsymbol{p},\boldsymbol{p+2}]_{\rm s}$ multiplets of $\SU(2\vert1,1)_{\rm L}\times \SU(2\vert1,1)_{\rm R}$, for $k\geq2$ and $p\geq3$~\cite{deBoer:1998kjm}. The $\SO(4)$ representations are given by a couple of $\SU(2)$ spins. The conjugate spin-2 multiplet $[\boldsymbol{p+2},\boldsymbol{p}]_{\rm s}$ is obtained by inverting $\Delta_{\rm L}$ with $\Delta_{\rm R}$, taking the opposite spin $-s$ and exchanging the $\SU(2)$ spins inside each $\SO(4)_{\rm gauged}$ and $\SO(4)_{\rm global}$ representations. Taken from ref.~\cite{Eloy:2020uix}.}
  \label{tab:multi20}
\end{table}


\paragraph{ExFT cubic Couplings}

TT term obtained from Cadabra + hand massaging
\begin{equation}
	\begin{split}
		TT = & - \frac{5}{32}T_{S T}\,^{\Sigma \Omega} T_{U V}\,^{\Lambda \Delta} c^{\Omega \Delta \Gamma} \delta^{T V} \phi^{S U \Sigma} \phi_{W}\,^{W \Gamma \Lambda}
		+T_{S T}\,^{\Sigma \Omega} T_{U V}\,^{\Lambda \Delta} c^{\Omega \Delta \Gamma} \delta^{T V} \phi^{S U \Lambda \Sigma \Gamma}\\
		& +\frac{3}{2}T_{S T}\,^{\Sigma \Omega} T_{U V}\,^{\Lambda \Sigma} c^{\Omega \Delta \Gamma} \delta^{T V} \phi^{U S \Delta \Gamma \Lambda} 
		+\frac{5}{2}T_{S T}\,^{\Sigma \Omega} T_{U V}\,^{\Lambda \Sigma} \phi^{T V \Delta} \phi^{S U \Lambda \Gamma} c^{\Omega \Delta \Gamma}
		-T_{S T}\,^{\Sigma \Omega} T_{U V}\,^{\Lambda \Delta} \phi^{S V \Gamma} \phi^{U T \Lambda \Sigma} c^{\Omega \Delta \Gamma} \\
		&-T_{S T}\,^{\Sigma \Omega} T_{U V}\,^{\Lambda \Sigma} \phi^{T U \Delta} \phi^{S V \Gamma \Lambda} c^{\Omega \Delta \Gamma} 
		- \frac{1}{2}T_{S T}\,^{\Sigma \Omega} T_{U V}\,^{\Lambda \Sigma} \phi_{W}\,^{U \Delta} \varphi^{S T \Lambda} \varphi^{V W \Gamma} c^{\Omega \Delta \Gamma} \\
		&+\frac{1}{2}T_{S T}\,^{\Sigma \Omega} T_{U V}\,^{\Lambda \Delta} \phi_{W}\,^{S \Sigma} \varphi^{T W \Lambda} \varphi^{U V \Gamma} c^{\Omega \Delta \Gamma}
		+\frac{1}{4}T_{S T}\,^{\Sigma \Omega} T_{U V}\,^{\Lambda \Sigma} \phi_{W}\,^{S \Delta} \varphi^{T W \Lambda} \varphi^{U V \Gamma} c^{\Omega \Delta \Gamma}
	\end{split}
\end{equation}

\begin{equation}
\begin{split}
TTCamille:=c^{\Sigma \Omega \Gamma}
&(\gamma T_{X Y}\,^{\Delta \Omega} T_{W V}\,^{\Lambda \Sigma}(\frac34 \Delta^{X W} j\,^{S T \Lambda} j_{S T}\,^{\Delta} j^{Y V \Gamma}
+\Delta^{X W} j^{Y S \Lambda} j_{S U}\,^{\Gamma} j^{U V \Delta} \\
&-j^{X U \Lambda} \Delta_{U}\,^{Y} j^{W S \Delta} j_{S}\,^{V \Gamma}
+\frac32 j^{X U \Lambda} \Delta_{U}\,^{Y} j^{W S \Gamma} j_{S}\,^{V \Delta}
-j^{Y V \Gamma} j^{X S \Lambda} \Delta_{S U} j^{U W \Delta})\\
&+\gamma T_{X Y}\,^{\Delta \Lambda} T_{W V}\,^{\Lambda \Sigma}(\Delta^{X W} j^{Y S \Gamma} j_{S U}\,^{\Delta} j^{U V \Omega}
-\Delta^{X W} j^{Y S \Delta} j_{S U}\,^{\Omega} j^{U V \Gamma}\\
&-2\Delta_{S U} j^{Y V \Omega} j^{X S \Gamma} j^{U W \Delta}
+2\Delta^{X W} j^{Y S \Omega} j_{S U}\,^{\Gamma} j^{U V \Delta}
-2\Delta_{S U} j^{Y V \Delta} j^{X S \Omega} j^{U W \Gamma}));
\end{split}
\end{equation}

XT terms 

\begin{equation}
	\begin{split}
		XTheta=c^{\Sigma \Omega \Gamma}&(-3 \Theta_{K L M N} \Delta^{K P} T_{P Q}\,^{\Delta \Sigma} j^{M U \Delta} \Delta_{U}\,^{N} j^{L R \Omega} \Delta_{R S} j^{S Q \Gamma}
		+6 \Theta_{K L M N} \Delta^{K P} T_{P Q}\,^{\Delta \Sigma} j^{M U \Gamma} j_{U}\,^{N \Delta} j^{L Q \Omega} \\
		&+2 \Theta_{K L M N} T_{P Q}\,^{\Delta \Sigma} j^{M U \Gamma} \Delta_{U}\,^{N} j^{K P \Delta} j^{L Q \Omega}
		-2 \Theta_{K L M N} T_{P Q}\,^{\Delta \Sigma} j^{M U \Delta} \Delta_{U}\,^{N} j^{K P \Gamma} j^{L Q \Omega}\\
		&-3 (\Theta_{K L}+\Theta \eta_{K L}) T_{P Q}\,^{\Delta \Sigma} j^{L Q \Delta} j^{K R \Omega} \Delta_{R S} j^{S P \Gamma}\\ 
		&+3 \Theta_{K L M N} (\eta^{K P} \eta^{L Q}- \Delta^{K P} \Delta^{L Q}) T_{P Q}\,^{\Delta \Sigma} \Delta_{R S} j^{M U \Delta} j_{U}\,^{R \Omega} j^{S N \Gamma});
	\end{split}
\end{equation}

XX terms ($\Theta \Theta$)


\bibliography{references}


\end{document}